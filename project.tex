\documentclass[12pt, a4paper, hidelinks]{scrartcl}

  % Enables the use of colour.
  \usepackage{xcolor}
  % Syntax high-lighting for code. Requires Python's pygments.
  \usepackage{minted}
  % Enables the use of umlauts and other accents.
  \usepackage[utf8]{inputenc}
  % Diagrams.
  \usepackage{tikz}
  % Settings for captions, such as sideways captions.
  \usepackage{caption}
  % Symbols for units, like degrees and ohms.
  \usepackage{gensymb}
  % Latin modern fonts - better looking than the defaults.
  \usepackage{lmodern}
  % Allows for columns spanning multiple rows in tables.
  \usepackage{multirow}
  % Better looking tables, including nicer borders.
  \usepackage{booktabs}
  % More math symbols.
  \usepackage{amssymb}
  % More math fonts, like \mathbb.
  \usepackage{amsfonts}
  % More math layouts, equation arrays, etc.
  \usepackage{amsmath}
  % More theorem environments.
  \usepackage{amsthm}
  % More column formats for tables.
  \usepackage{array}
  % Adjust the sizes of box environments.
  \usepackage{adjustbox}
  % Better looking single quotes in verbatim and minted environments.
  \usepackage{upquote}
  % Better blank space decisions.
  \usepackage{xspace}
  % Better looking tikz trees.
  \usepackage{forest}
  % URLs.
  \usepackage{hyperref}
  % Plotting.
  \usepackage{pgfplots}
  % Changing headers and footers.
  \usepackage{fancyhdr}
  % Calculates the number of pages.
  \usepackage{lastpage}
  % Styling the abstract.
  \usepackage{abstract}
  
  % Various tikz libraries.
  % For drawing mind maps.
  \usetikzlibrary{mindmap}
  % For adding shadows.
  \usetikzlibrary{shadows}
  % Extra arrows tips.
  \usetikzlibrary{arrows.meta}
  % Old arrows.
  \usetikzlibrary{arrows}
  % Automata.
  \usetikzlibrary{automata}
  % For more positioning options.
  \usetikzlibrary{positioning}
  % Creating chains of nodes on a line.
  \usetikzlibrary{chains}
  % Fitting node to contain set of coordinates.
  \usetikzlibrary{fit}
  % Extra shapes for drawing.
  \usetikzlibrary{shapes}
  % For markings on paths.
  \usetikzlibrary{decorations.markings}
  % For advanced calculations.
  \usetikzlibrary{calc}
  
  % GMIT colours.
  \definecolor{gmitblue}{RGB}{20,134,225}
  \definecolor{gmitred}{RGB}{220,20,60}
  \definecolor{gmitgrey}{RGB}{67,67,67}

  %%%%%% CHANGE
  % Name of the module.
  \newcommand{\modulename}{Programming and Scripting}
  % Year of delivery.
  \newcommand{\projectyear}{2018}
  % Title of project.
  \newcommand{\projectname}{Project \projectyear}
  % Due date.
  \newcommand{\duedate}{last commit on or before April 29\textsuperscript{th}}
  %%%%%%

  % Abstract
  \renewcommand{\abstractname}{}
  \renewcommand{\absnamepos}{empty}

  % Header and footer
  \fancypagestyle{plain}{
    \renewcommand{\headrulewidth}{0pt}
    \lhead{}
    \rhead{}
    \cfoot{Page~\thepage~of~\pageref{LastPage}}
    \renewcommand{\footrulewidth}{0.1pt}
  }
  \pagestyle{fancy}
  \fancyhf{}
  \lhead{\modulename}
  \rhead{\projectname}
  \cfoot{Page~\thepage~of~\pageref{LastPage}}
  \renewcommand{\headrulewidth}{0.1pt}
  \renewcommand{\footrulewidth}{0.1pt}

  % Tables
  \newcolumntype{x}[1]{>{\centering\arraybackslash\hspace{0pt}}p{#1}}

  % Bibliography
  \renewcommand{\refname}{\selectfont\normalsize References} 

  \title{\projectname}
  \author{\modulename}
  \date{Due: \duedate}

\begin{document}
  
\maketitle

\noindent
This document contains the instructions for \projectname{} for \modulename{}.
Note that you are not expected to know how to do the whole project from the beginning.
Rather, it is expected that you will research ways to tackle the project and then formulate your own submission based on your investigations.
Please be advised that all students are bound by the Quality Assurance Framework~\cite{gmitqaf} at GMIT which includes the Code of Student Conduct and the Policy on Plagiarism.

\subsection*{Problem statement}
  The following project concerns the well-known Fisher's Iris data set~\cite{irisdataset}.
  The project entails you researching the data set, and then writing documentation and code in the Python programming language~\cite{pythonwebsite} based on that research.
  
  An online search for information on the data set will convince you that many people have investigated and written about it previously, and many of those are not experienced programmers.
  You are expected to be able to break this project into several smaller tasks that are easier to solve, and to plug these together after they have been completed.
  You might do that for this project as follows:
  \begin{enumerate}
    \item Research background information about the data set and write a summary about it.
    \item Keep a list of references you used in completing the project.
    \item Download the data set and write some Python code to investigate it.
    \item Summarise the data set by, for example, calculating the maximum, minimum and mean of each column of the data set. A Python script will quickly do this for you.
    \item Write a summary of your investigations.
    \item Include supporting tables and graphics as you deem necessary.
  \end{enumerate}

  It might help to suppose that your manager at work has asked you to investigate this data set, with a view to explaining it to your work colleagues.
  Imagine that you are to give a brief presentation on the data set to them in a few weeks' time, where you explain to them what investigating a data set entails and how Python can be used to do it.
  You have been asked not to create a deck of presentation slides, but rather to present your write-up and code to them.


\subsection*{Minimum Viable Project}
  The minimum standard for this project is a GitHub repository containing a README and a Python script.
  The README should contain a summary of the data set and your investigations into it.
  It should also clearly document how to run the Python code you used to investigate the data set, and what that code does.
  Furthermore, it should list all references used in completing the project.

  A better project will be well organised and contain detailed explanations.
  The analysis will be well conceived, and examples of interesting analyses that others have pursued based on the data set will be discussed.
  
  Note that the point of this project is to use Python.
  You may use any Python libraries that you wish, whether they have been discussed in class or not.
  You should not be thinking of using spreadsheet software like Excel to do your calculations.

\subsection*{Submissions}
  GitHub must be used to manage your project submission.
  Your GitHub repository will form the main submission of the project.
  You must submit the URL of your GitHub repository using the link on the course Moodle page before the deadline.
  You can do this at any time, as the last commit before the deadline will be used as your submission for this project.

  Any submission that does not have a full and incremental git history with informative commit messages over the course of the project timeline will be accorded a proportionate mark.
  It is expected that your repository will have lots of commits, with each commit relating to a reasonably small unit of work.
  In the last week of term, or at any other time, you may be asked by the lecturer to explain the contents of your git repository.
  While it is encouraged that students will engage in peer learning, any unreferenced documentation and software that is contained in your submission must have been written by you.
  You can show this by having an incremental commit history and by being able to explain your code.

\subsection*{Marking scheme}
  This project will be worth 50\% of your mark for this module.
  The following marking scheme will be used to mark the project out of 100\%.
  Students should note, however, that in certain circumstances the examiner's overall impression of the project may influence marks in each individual component.

  \begin{center}
    \begin{tabular}{llp{8.4cm}}
      \toprule
      25\% & \textbf{Research} & Investigation of each notebook and script as demonstrated by references, background information, and approach. \\
      \midrule
      25\% & \textbf{Development} & Clear, well-written, and efficient code with appropriate comments. \\
      \midrule
      25\% & \textbf{Consistency} & Good planning and pragmatic attitude to work as evidenced by commit history. \\
      \midrule
      25\% & \textbf{Documentation} & Concise descriptions and explanations of theoretical and practical aspects of problems. \\
      \bottomrule
    \end{tabular}
  \end{center}

\subsection*{Advice for students}
  \begin{itemize}
    \item
      Your git commit history should be extensive.
      A reasonable unit of work for a single commit is a small function, or a handful of comments, or a small change that fixes a bug.
      If you are well organised you will find it easier to determine the size of a reasonable commit, and it will show in your git history.
    \item
      Using information, code and data from outside sources is sometimes acceptable --- so long as it is licensed to permit this, you clearly reference the source, and the overall project is substantially your own work.
      Using a source that does not meet these three conditions could jeopardise your mark.
    \item
      You must be able to explain your project during and after its completion.
      Bear this in mind when you are writing your README.
      If you had trouble understanding something in the first place, you will likely have trouble explaining it a couple of weeks later.
      Write a short explanation of it in your README, so that you can jog your memory later.
    \item
      Everyone is susceptible to procrastination and disorganisation.
      You are expected to be aware of this and take reasonable measures to avoid them.
      The best way to do this is to draw up an initial straight-forward project plan and keep it updated.
      You can show the examiner that you have done this in several ways.
      The easiest is to summarise the project plan in your README.
      Another way is to use a to-do list like GitHub Issues.
    \item
      Students have problems with projects from time to time.
      Some of these are unavoidable, such as external factors relating to family issues or illness.
      In such cases allowances can sometimes be made.
      Other problems are preventable, such as missing the submission deadline because you are having internet connectivity issues five minutes before it.
      Students should be able to show that up until an issue arose they had completed a reasonable and proportionate amount of work, and took reasonable steps to avoid preventable issues.
    \item
      Go easy on yourself - this is one project in one module.
      It will not define you or your life.
      A higher overall course mark should not be determined by a single project, but rather your performance in all your work in all your modules.
      Here, you are just trying to demonstrate to yourself, to the examiners, and to prospective future employers, that you can take a reasonably straight-forward problem and solve it within a few weeks.
  \end{itemize}


  \bibliographystyle{plain}
  \bibliography{bibliography}
\end{document}
